% !TeX root = ../main_problemset.tex
\documentclass[../main_problemset.tex]{subfiles} % Inherits definitions from parent .tex file.

% Per-problem variable definitions
\newcommand{\problemName}{A. Organisasi Kemahasiswaan}
\newcommand{\problemTL}{2 s}
\newcommand{\problemML}{256 MB}

\begin{document}

\begin{center}
    \section*{\problemName}
    \addcontentsline{toc}{section}{\problemName} % for pdf indexing
    
    \begin{tabular}{rl}
    Batasan waktu : & \problemTL \\
    Batasan memori : & \problemML
    \end{tabular}
\end{center}

\subsection*{Deskripsi}
\addcontentsline{toc}{subsection}{Deskripsi} % for pdf indexing

Terdapat N organisasi kemahasiswaan di Fasilkom UI. Organisasi ke-i terdiri atas tepat M[i] mahasiswa. Menurut peraturan fakultas, setiap mahasiswa hanya boleh tergabung pada paling banyak K organisasi.

Berapakah banyaknya mahasiswa Fasilkom UI paling sedikit yang mungkin?

\subsection*{Format Masukan}
\addcontentsline{toc}{subsection}{Format Masukan} % for pdf indexing

Baris pertama berisi sebuah bilangan bulat T yang menyatakan banyaknya kasus uji. Baris-baris berikutnya berisi T kasus uji, yang masing-masing diberikan dalam format berikut ini:

\begin{lcverbatim}
N K
M[1] M[2] .. M[N]
\end{lcverbatim}

\subsection*{Format Keluaran}
\addcontentsline{toc}{subsection}{Format Keluaran} % for pdf indexing

Untuk setiap kasus uji, keluarkan sebuah baris berisi banyaknya mahasiswa Fasilkom UI paling sedikit yang mungkin.

\vspace{.4cm}

\begin{minipage}[t]{0.5\textwidth}
\subsection*{Contoh Masukan}
\addcontentsline{toc}{subsection}{Contoh Masukan} % for pdf indexing

\begin{lcverbatim}
3
2 2
10 12
2 1
10 12
3 2
3 2 1
\end{lcverbatim}
\end{minipage}
\begin{minipage}[t]{0.5\textwidth}
\subsection*{Contoh Keluaran}
\addcontentsline{toc}{subsection}{Contoh Keluaran} % for pdf indexing

\begin{lcverbatim}
12
22
3
\end{lcverbatim}

\end{minipage}

\textit{Perhatikan bahwa contoh ketiga tidak termasuk dalam contoh masukan dan contoh keluaran dari soal versi mudah.}

\subsection*{Penjelasan}
\addcontentsline{toc}{subsection}{Penjelasan} % for pdf indexing

Misalkan mahasiswa-mahasiswa dinomori dengan bilangan bulat positif.

Untuk contoh pertama, salah satu struktur keanggotaan organisasi yang mungkin adalah:
\begin{itemize}
	\item anggota-anggota organisasi 1: \{1, 2, 3, 4, 5, 6, 7, 8, 9, 10\}
	\item anggota-anggota organisasi 2: \{1, 2, 3, 4, 5, 6, 7, 8, 9, 10, 11, 12\}
\end{itemize}


Untuk contoh kedua, salah satu struktur keanggotaan organisasi yang mungkin adalah:

\begin{itemize}
	\item anggota-anggota organisasi 1: \{1, 2, 3, 4, 5, 6, 7, 8, 9, 10\}
	\item anggota-anggota organisasi 2: \{11, 12, 13, 14, 15, 16, 17, 18, 19, 20, 21, 22\}
\end{itemize}

Untuk contoh ketiga, salah satu struktur keanggotaan organisasi yang mungkin adalah:

\begin{itemize}
	\item anggota-anggota organisasi 1: \{1, 2, 3\}
	\item anggota-anggota organisasi 2: \{1, 3\}
	\item anggota-anggota organisasi 3: \{2\}
\end{itemize}

\subsection*{Batasan}
\addcontentsline{toc}{subsection}{Batasan} % for pdf indexing

\begin{minipage}[t]{0.47\textwidth}

Batasan yang berlaku untuk versi mudah dan versi sulit:

\begin{itemize}
	\item 1 $ \leq $ T $ \leq $ 10
	\item 1 $ \leq $ K $ \leq $ N
	\item 1 $ \le $ M[i] $ \le $ 100.000
\end{itemize}
\end{minipage}
\begin{minipage}[t]{0.06\textwidth}
	\hfill
\end{minipage}
\begin{minipage}[t]{0.47\textwidth}
	Batasan khusus versi mudah:
	\begin{itemize}
		\item 1 $ \le $ N $ \le $ 2
	\end{itemize}
	
	\vspace{.2cm}
	
	Batasan khusus versi sulit:
	\begin{itemize}
		\item 1 $ \le $ N $ \le $ 100.000
	\end{itemize}
\end{minipage}

\end{document}
