% !TeX root = ../main_problemset.tex
\documentclass[../main_problemset.tex]{subfiles} % Inherits definitions from parent .tex file.

% Per-problem variable definitions
\newcommand{\problemName}{B. Fotografer Wisuda}
\newcommand{\problemTL}{1 s}
\newcommand{\problemML}{64 MB}

\begin{document}

\begin{center}
    \section*{\problemName}
    \addcontentsline{toc}{section}{\problemName} % for pdf indexing
    
    \begin{tabular}{rl}
    Batasan waktu : & \problemTL \\
    Batasan memori : & \problemML
    \end{tabular}
\end{center}

\subsection*{Deskripsi}
\addcontentsline{toc}{subsection}{Deskripsi} % for pdf indexing

Pak Chanek sedang mengunjungi perayaan wisuda teman-temannya di Balairung Universitas Indonesia. Seperti biasa, ia sering diminta untuk memotret teman-temannya di halaman Balairung.

Kali ini, terdapat $ A $ teman laki-laki dan $ B $ teman perempuan Pak Chanek yang meminta untuk difoto. Bosan dengan pose foto pada umumnya, Pak Chanek ingin menjejerkan mereka dalam sebuah barisan lurus yang memenuhi seluruh syarat berikut ini:
\begin{itemize}
	\item Tidak ada subbarisan yang terdiri atas lebih dari 2 teman laki-laki yang bersebelahan.
	\item Tidak ada subbarisan yang terdiri atas lebih dari $ K $ teman perempuan yang bersebelahan.
\end{itemize}

Bantulah Pak Chanek untuk mencarikan sebuah barisan yang memenuhi syarat-syarat tersebut, atau laporkan apabila hal tersebut mustahil.

\subsection*{Format Masukan}
\addcontentsline{toc}{subsection}{Format Masukan} % for pdf indexing

Baris pertama berisi sebuah bilangan bulat $ T $ yang menyatakan banyaknya kasus uji. Baris-baris berikutnya berisi $ T $ kasus uji, yang masing-masing diberikan dalam format berikut ini:

\begin{lcverbatim}
A B K
\end{lcverbatim}

\subsection*{Format Keluaran}
\addcontentsline{toc}{subsection}{Format Keluaran} % for pdf indexing

Untuk setiap kasus uji, keluarkan sebuah baris berisi sebuah string yang terdiri atas $ A + B $ karakter yang menyatakan sebuah barisan yang memenuhi seluruh syarat. Nyatakan teman laki-laki dengan karakter \texttt{L}, dan teman perempuan dengan karakter \texttt{P}.

Apabila terdapat lebih dari satu barisan yang mungkin, \textbf{keluarkan yang mana saja}.

Apabila tidak ada barisan yang mungkin, keluarkan \texttt{mustahil}.

\vspace{.4cm}

\begin{minipage}[t]{0.5\textwidth}
\subsection*{Contoh Masukan}
\addcontentsline{toc}{subsection}{Contoh Masukan} % for pdf indexing

\begin{lcverbatim}
4
0 4 4
0 4 3
1 3 3
3 3 3
\end{lcverbatim}
\end{minipage}
\begin{minipage}[t]{0.5\textwidth}
\subsection*{Contoh Keluaran}
\addcontentsline{toc}{subsection}{Contoh Keluaran} % for pdf indexing

\begin{lcverbatim}
PPPP
mustahil
LPPP
LPPLPL
\end{lcverbatim}
\end{minipage}

\subsection*{Penjelasan}
\addcontentsline{toc}{subsection}{Penjelasan} % for pdf indexing

Pada contoh kedua, satu-satunya barisan yang mungkin, \texttt{PPPP}, merupakan barisan yang tidak diperbolehkan karena terdapat 4 (yakni, lebih dari $ K=3 $) teman perempuan yang bersebelahan.

Pada contoh keempat, \texttt{LLLPPP} merupakan barisan yang tidak diperbolehkan karena terdapat 3 (yakni, lebih dari $ K=2 $) teman laki-laki yang saling bersebelahan. Perhatikan bahwa terdapat beberapa barisan lain yang diperbolehkan, misalnya \texttt{PLLPPL}.

\pagebreak
\subsection*{Batasan}
\addcontentsline{toc}{subsection}{Batasan} % for pdf indexing

Batasan yang berlaku untuk versi mudah dan versi sulit:

\begin{itemize}
	\item $ 1 \le T \le 20 $
	\item $ 1 \le B \le 1.000 $
	\item $ 1 \le K \le B $
\end{itemize}

Batasan khusus versi mudah:

\begin{itemize}
	\item $0 \le A \le 1$
\end{itemize}

Batasan khusus versi sulit:

\begin{itemize}
	\item $0 \le A \le 1.000$
\end{itemize}

\end{document}
