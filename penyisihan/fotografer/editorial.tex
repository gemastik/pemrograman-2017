% !TeX root = ../main_editorial.tex
\documentclass[../main_editorial.tex]{subfiles} % Inherits definitions from parent .tex file.

% Per-problem variable definitions
\newcommand{\problemName}{Fotografer Wisuda}
\newcommand{\problemWriter}{Ashar Fuadi}
\newcommand{\problemEditorialWriter}{Anthony}
\newcommand{\problemTags}{algoritma konstruktif}

\begin{document}

\begin{center}
    \section*{\problemName}
    \addcontentsline{toc}{section}{\problemName} % for pdf indexing
    
    \begin{tabular}{rl}
    Penulis soal : & \problemWriter \\
    Penulis editorial : & \problemEditorialWriter \\
    Tema : & \problemTags
    \end{tabular}
\end{center}

\subsection*{Catatan/Komentar}
\addcontentsline{toc}{subsection}{Catatan/Komentar} % for pdf indexing

Ide untuk menyelesaikan soal ini tidak sulit tetapi cukup rawan terjadi kesalahan pada implementasinya.

Batasan kedua versi: $1 \le K \le B \le 1.000$

\subsection*{Versi Mudah}
\addcontentsline{toc}{subsection}{Versi Mudah} % for pdf indexing
Batasan: $\mathbf{0 \le A \le 1}$

Untuk menyelesaikan versi mudah dari soal ini, penyelesaian untuk seluruh kemungkinan masukan dapat dikategorikan sebagai berikut:

\begin{itemize}
	\item Jika tidak ada teman laki-laki ($A = 0$):
		\begin{itemize}
			\item jika banyaknya teman perempuan melebihi $K$ ($B > K$), jawabannya adalah \texttt{mustahil},
			\item jika tidak, cetak \texttt{P} sebanyak $B$ kali.
		\end{itemize}
	\item Jika terdapat tepat 1 teman laki-laki ($A = 1$):
		\begin{itemize}
			\item jika banyaknya teman perempuan melebihi $2K$ ($B > 2K$), jawabannya adalah \texttt{mustahil},
			\item selain itu, cetak \texttt{P} sebanyak $\min(B, K)$, kemudian \texttt{L}, kemudian \texttt{P} sebanyak $B - \min(B, K)$.
		\end{itemize}
\end{itemize}

\subsection*{Versi Sulit}
\addcontentsline{toc}{subsection}{Versi Sulit} % for pdf indexing

Batasan: $\mathbf{0 \le A \le 1.000}$

Hal pertama yang dapat dilakukan adalah pemeriksaan kasus \texttt{mustahil}, yaitu jika tidak memungkinkan untuk menempatkan teman laki-laki sehingga paling banyak dua teman laki-laki bersebelahan untuk setiap deretan teman laki-laki. Dengan kata lain, kasus \texttt{mustahil} hanya terjadi pada masukan yang memenuhi:
\begin{itemize}
	\item $A > 2(B + 1)$ atau
	\item $A < \lceil \frac{B}{K} \rceil - 1$.
\end{itemize}

Perhatikan bahwa $\lceil \frac{B}{K} \rceil - 1$ dan $2(B + 1)$ berturut-turut merupakan batas bawah dan batas atas banyaknya deretan teman perempuan (\texttt{P}) yang dipisahkan oleh deretan teman laki-laki (\texttt{L}).

Untuk masukan yang tidak \texttt{mustahil}, solusi dapat dibangun dengan cara menemukan pola "penyelipan" $A$ buah \texttt{L} ke dalam deretan yang terdiri dari $B$ buah \texttt{P} sedemikian rupa sehingga memenuhi persyaratan yang diberikan pada soal. Sebagai contoh, pola yang dapat dibentuk adalah \texttt{P...PLLP...PLLP...PLP...PLP...P}.

\end{document}
