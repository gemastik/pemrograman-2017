\documentclass[a4paper]{article}
\usepackage[margin=1in]{geometry}

\usepackage{graphicx}
\usepackage{amsmath}
\usepackage{amsthm}
\usepackage{amssymb}
\usepackage{float}
\usepackage{listings}
\usepackage{mathtools}
\usepackage{subfiles}
\usepackage{hyperref}
\usepackage{tikz}
\usepackage{caption}
\usepackage{pgfplots}

\newtheorem{algorithm_def}{Algorithm}[subsection]
\newtheorem{definition}{Definition}[subsection]
\newtheorem{lemma}{Lemma}[subsection]
\newtheorem{theorem}{Theorem}[subsection]
\newtheorem{corollary}{Corollary}[subsection]

\setlength{\parindent}{0pt}
\setlength{\parskip}{6pt}

\usepackage{enumitem}
\setlist{nolistsep}

\hyphenation{
	me-nge-ta-hu-i
	kon-fi-gu-ra-si
	me-nge-na-i
}

\begin{document}

\begin{titlepage}
\begin{center}

    \textbf{\huge BUKLET PEMBAHASAN SOAL}
    
    \vspace{1cm} % padding
    
    \includegraphics[width=0.8\textwidth]{../gemastiklogo}
    
	\vspace{1cm} % padding
    
    \textbf{\LARGE PENYISIHAN PEMROGRAMAN GEMASTIK 10}\\[0.5cm]
    \textbf{\LARGE 30 September 2017}

	\vfill
	
	{\Large Soal dan Penulis Soal}
	
	\vspace{.1cm}
	
	\def\arraystretch{1.5} % biar megar
	\begin{tabular}{|c|l|l|}
		\hline
		\textbf{Kode} & \textbf{Judul} & \textbf{Penulis} \\
		\hline
		A & Berbalas Pantun & Alham Fikri Aji \\
		B & Fotografer Wisuda & Ashar Fuadi \\
		C & Penulis Soal & Ashar Fuadi \\
		D & Saklar Lhompat II & Anthony \\
		E & Pasangan Terbaik & Alham Fikri Aji \\
		F & Rubrik Petakata & Ashar Fuadi \\
		\hline
	\end{tabular}
	
	\vspace{.3cm}
	
	{\Large Penguji}
	
	Ammar Fathin Sabili

\end{center}
\end{titlepage}

\section*{Statistik Penyisihan}
\addcontentsline{toc}{section}{Statistik Umum} % for pdf indexing

Catatan: Statistik dibuat berdasarkan kondisi sebelum tabel peringkat dibekukan.

\begin{itemize}
\item Penyisihan Pemrograman GEMASTIK 10 diikuti oleh 316 tim, dengan 233 tim setidaknya mengirimkan satu pengumpulan jawaban.

\vspace*{.2cm}
\item Statistik banyaknya soal yang diselesaikan oleh seluruh tim:

\begin{table}[H]
\centering
\begin{tabular}{|l|c|c|c|c|c|c|c|c|c|c|c|c|c|}
\hline
\textbf{\# \textit{Accepted}} & 0  & 1  & 2  & 3  & 4  & 5 & 6 & 7 & 8 & 9 & 10 & 11 & 12 \\ \hline
\textbf{\# Tim}      & 31 & 67 & 39 & 60 & 11 & 9 & 2 & 3 & 4 & 3 & 2  & 0  & 2  \\ \hline
\end{tabular}
\end{table}

\vspace*{.2cm}
\item Statistik per soal:

\begin{table}[H]
\centering
\begin{tabular}{l|c|c|c|c|c|c|c|c|c|c|c|c|}
\cline{2-13}
                                                        & \multicolumn{2}{c|}{\textbf{A}} & \multicolumn{2}{c|}{\textbf{B}} & \multicolumn{2}{c|}{\textbf{C}} & \multicolumn{2}{c|}{\textbf{D}} & \multicolumn{2}{c|}{\textbf{E}} & \multicolumn{2}{c|}{\textbf{F}} \\ \cline{2-13} 
                                                        & \textbf{1}     & \textbf{2}     & \textbf{1}     & \textbf{2}     & \textbf{1}     & \textbf{2}     & \textbf{1}     & \textbf{2}     & \textbf{1}     & \textbf{2}     & \textbf{1}     & \textbf{2}     \\ \hline
\multicolumn{1}{|l|}{\textbf{\# \textit{Accepted}}}          & 199            & 121            & 113            & 28             & 25             & 14             & 15             & 7              & 18             & 5              & 3              & 2              \\ \hline
\multicolumn{1}{|l|}{\textbf{\# Pengiriman Solusi}} & 384            & 653            & 459            & 209            & 262            & 56             & 53             & 24             & 335            & 41             & 9              & 4              \\ \hline
\end{tabular}
\end{table}

Pertama untuk mendapatkan \textit{Accepted}:

\begin{itemize}
\item Soal A1:
\begin{itemize}
\item TMCLL (Institut Teknologi Sepuluh Nopember)
\item pada menit ke-2
\end{itemize}

\item Soal A2:
\begin{itemize}
\item gak ada pencerahan (Universitas Indonesia)
\item pada menit ke-3
\end{itemize}

\item Soal B1:
\begin{itemize}
\item Fata Ganteng (Universitas Indonesia)
\item pada menit ke-10
\end{itemize}

\item Soal B2:
\begin{itemize}
\item Fata Ganteng (Universitas Indonesia)
\item pada menit ke-22
\end{itemize}

\item Soal C1:
\begin{itemize}
\item Ainge ST (Institut Teknologi Bandung)
\item pada menit ke-16
\end{itemize}

\item Soal C2:
\begin{itemize}
\item Ainge ST (Institut Teknologi Bandung)
\item pada menit ke-16
\end{itemize}

\item Soal D1:
\begin{itemize}
\item collateral\_damage (Universitas Bina Nusantara)
\item pada menit ke-33
\end{itemize}

\item Soal D2:
\begin{itemize}
\item gak ada pencerahan (Universitas Indonesia)
\item pada menit ke-61
\end{itemize}

\item Soal E1:
\begin{itemize}
\item gak ada pencerahan (Universitas Indonesia)
\item pada menit ke-15
\end{itemize}

\item Soal E2:
\begin{itemize}
\item gak ada pencerahan (Universitas Indonesia)
\item pada menit ke-26
\end{itemize}

\item Soal F1:
\begin{itemize}
\item gak ada pencerahan (Universitas Indonesia)
\item pada menit ke-83
\end{itemize}

\item Soal F2:
\begin{itemize}
\item gak ada pencerahan (Universitas Indonesia)
\item pada menit ke-83
\end{itemize}

\end{itemize}

\end{itemize}

\pagebreak

\subfile{pantun/editorial}
\pagebreak
\subfile{fotografer/editorial}
\pagebreak
\subfile{sc/editorial}
\pagebreak
\subfile{lhompat2/editorial}
\pagebreak
\subfile{pairing/editorial}
\pagebreak
\subfile{scrabble/editorial}

\end{document}
