\documentclass[a4paper]{article}
\usepackage[margin=1in]{geometry}

\usepackage{graphicx}
\usepackage{amsmath}
\usepackage{amsthm}
\usepackage{listings}
\usepackage{mathtools}
\usepackage{subfiles}
\usepackage{hyperref}

\newtheorem{algorithm_def}{Algorithm}[subsection]
\newtheorem{definition}{Definition}[subsection]
\newtheorem{lemma}{Lemma}[subsection]
\newtheorem{theorem}{Theorem}[subsection]
\newtheorem{corollary}{Corollary}[subsection]

\setlength{\parindent}{0pt}
\setlength{\parskip}{6pt}

\usepackage{enumitem}
\setlist{nolistsep}

\begin{document}

\begin{titlepage}
\begin{center}

    \textbf{\huge BUKLET PEMBAHASAN SOAL}
    
    \vspace{1cm} % padding
    
    \includegraphics[width=0.8\textwidth]{../gemastiklogo}
    
	\vspace{1cm} % padding
    
    \textbf{\LARGE PENYISIHAN PEMROGRAMAN GEMASTIK 10}\\[0.5cm]
    \textbf{\LARGE 30 September 2017}

	\vfill
	
	{\Large Soal-Soal}
	
	\def\arraystretch{1.5} % biar megar
	\begin{tabular}{|c|l|l|}
		\hline
		\textbf{Kode} & \textbf{Judul} & \textbf{Penulis} \\
		\hline
		A & Berbalas Pantun & Alham Fikri Aji \\
		B & Fotografer Wisuda & Ashar Fuadi \\
		C & Penulis Soal & Ashar Fuadi \\
		D & Saklar Lhompat II & Anthony \\
		E & Pasangan Terbaik & Alham Fikri Aji \\
		F & Rubrik Petakata & Ashar Fuadi \\
		\hline
	\end{tabular}

\end{center}
\end{titlepage}

\subfile{pantun/editorial}
\pagebreak
\subfile{fotografer/editorial}
\pagebreak
\subfile{sc/editorial}
\pagebreak
\subfile{lhompat2/editorial}
\pagebreak
\subfile{pairing/editorial}
\pagebreak
\subfile{scrabble/editorial}

\end{document}
