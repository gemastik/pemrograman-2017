% !TeX root = ../main_problemset.tex
\documentclass[../main_problemset.tex]{subfiles} % Inherits definitions from parent .tex file.

% Per-problem variable definitions
\newcommand{\problemName}{E. Pasangan Terbaik}
\newcommand{\problemTL}{3 s}
\newcommand{\problemML}{128 MB}

\begin{document}

\begin{center}
    \section*{\problemName}
    \addcontentsline{toc}{section}{\problemName} % for pdf indexing
    
    \begin{tabular}{rl}
    Batasan waktu : & \problemTL \\
    Batasan memori : & \problemML
    \end{tabular}
\end{center}

\subsection*{Deskripsi}
\addcontentsline{toc}{subsection}{Deskripsi} % for pdf indexing

Pak Chanek memiliki tiga buah \textit{array} bilangan bulat: $ A $ dan $ B $ yang masing-masing memiliki $ N $ elemen, dan $ C $ yang memiliki $ M $ elemen. Untuk kemudahan, anggap bahwa indeks elemen-elemen \textit{array} dimulai dari $ 0 $.

Diberikan sepasang bilangan bulat $ i $ dan $ j $, Pak Chanek mendefinisikan $ f(i,j) $ sebagai berikut:

$$
f(i,j)=A[i]\times B[j]+C[(A[i]\times B[j])\bmod M]
$$

Pak Chanek ingin mengetahui nilai dari:

$$
\min_{0 \le i,j < N} f(i,j)
$$

atau dengan kata lain, nilai terkecil dari $ f(i,j) $ untuk semua kemungkinan pasangan indeks \textit{array} $ A $ dan $ B $.

Bantulah Pak Chanek menghitung nilai tersebut!

\subsection*{Format Masukan}
\addcontentsline{toc}{subsection}{Format Masukan} % for pdf indexing

Baris pertama berisi sebuah bilangan bulat $ T $ yang menyatakan banyaknya kasus uji. Baris-baris berikutnya berisi $ T $ kasus uji, yang masing-masing diberikan dalam format berikut ini:

\begin{lcverbatim}
N M
A[0] A[1] .. A[N-1]
B[0] B[1] .. B[N-1]
C[0] C[1] .. C[M-1]
\end{lcverbatim}

\subsection*{Format Keluaran}
\addcontentsline{toc}{subsection}{Format Keluaran} % for pdf indexing

Untuk setiap kasus uji, keluarkan sebuah baris berisi nilai dari $ \min_{0 \le i,j < N} f(i,j) $.

\vspace{.4cm}

\begin{minipage}[t]{0.5\textwidth}
\subsection*{Contoh Masukan}
\addcontentsline{toc}{subsection}{Contoh Masukan} % for pdf indexing

\begin{lcverbatim}
2
4 2
3 14 15 9
26 53 58 97
93 2
4 3
3 14 15 9
26 53 58 97
93 2 38
\end{lcverbatim}
\end{minipage}
\begin{minipage}[t]{0.5\textwidth}
\subsection*{Contoh Keluaran}
\addcontentsline{toc}{subsection}{Contoh Keluaran} % for pdf indexing

\begin{lcverbatim}
161
171
\end{lcverbatim}
\end{minipage}

\pagebreak
\subsection*{Penjelasan}
\addcontentsline{toc}{subsection}{Penjelasan} % for pdf indexing

Pada contoh pertama, pasangan $ (i,j) $ yang menyebabkan $ f(i,j) $ minimum adalah $ (i=0,j=1) $. Nilai dari $ f(0,1) $ adalah:

$$
A[0]\times B[1]+C[(A[0]\times B[1])\bmod 2]=3\times 53+C[(3\times 53)\bmod 2]=159+2=161
$$

Pada contoh kedua, pasangan $ (i,j) $ yang menyebabkan $ f(i,j) $ minimum adalah $ (i=0,j=0) $. Nilai dari $ f(0,0) $ adalah:

$$
A[0]\times B[0]+C[(A[0]\times B[0])\bmod 3]=3\times 26+C[(3\times 26)\bmod 3]=78+93=171
$$

\subsection*{Batasan}
\addcontentsline{toc}{subsection}{Batasan} % for pdf indexing

\begin{minipage}[t]{0.47\textwidth}

Batasan yang berlaku untuk versi mudah dan versi sulit:

\begin{itemize}
	\item $ 1 \le T \le 5 $
	\item $ 1 \le N \le 100.000 $
	\item $ 1 \le A[i], B[i], C[i] \le 1.000.000 $
\end{itemize}
\end{minipage}
\begin{minipage}[t]{0.06\textwidth}
    \hfill
\end{minipage}
\begin{minipage}[t]{0.47\textwidth}
Batasan khusus versi mudah:

\begin{itemize}
	\item $0 \le M \le 1.000$
\end{itemize}

\vspace{.2cm}

Batasan khusus versi sulit:

\begin{itemize}
	\item $0 \le M \le 100.000$
\end{itemize}
\end{minipage}

\end{document}
