% !TeX root = ../main_editorial.tex
\documentclass[../main_editorial.tex]{subfiles} % Inherits definitions from parent .tex file.

% Per-problem variable definitions
\newcommand{\problemName}{Pasangan Terbaik}
\newcommand{\problemWriter}{Alham Fikri Aji}
\newcommand{\problemEditorialWriter}{William Gozali}
\newcommand{\problemTags}{matematika, heuristik}

\newcommand{\Mod}[1]{\ \mathrm{mod}\ #1}
\newcommand{\bigO}[1]{\mathcal{O}(#1)}
\newcommand{\maxValue}{\mathcal{Z}}

\begin{document}

\begin{center}
    \section*{\problemName}
    \addcontentsline{toc}{section}{\problemName} % for pdf indexing
    
    \begin{tabular}{rl}
    Penulis soal : & \problemWriter \\
    Penulis editorial : & \problemEditorialWriter \\
    Tema : & \problemTags
    \end{tabular}
\end{center}

\subsection*{Catatan/Komentar}
\addcontentsline{toc}{subsection}{Catatan/Komentar} % for pdf indexing

Berikut adalah rumus yang diberikan pada soal.
$$f(i, j) = A[i] \times B[j] + C[(A[i] \times  B[j]) \Mod{M}]$$

Bagian penjumlahan dengan suatu elemen pada array $C$ merupakan bagian yang paling sulit dikendalikan. Sebab elemen $C$ yang dipilih memiliki pola yang sulit ditebak. Solusi yang digunakan akan fokus pada menganalisis bagian tersebut.

Untuk kemudahan, definisikan nilai terbesar yang mungkin untuk elemen array adalah $\maxValue$, dengan $\maxValue = 1.000.000$.

\subsection*{Versi Mudah}
\addcontentsline{toc}{subsection}{Versi Mudah} % for pdf indexing
Batasan: $1 \le M \le 1.000$

Perhatikan rumus indeks array $C$ yang akan dipilih. Berdasarkan sifat modulo, diketahui:
$$(A[i] \times B[j]) \Mod{M} = ((A[i]\Mod{M}) \times (B[j]\Mod{M})) \Mod{M}$$

Berapapun nilai $A[i]$, yang menentukan indeks array $C$ yang digunakan adalah nilai modulonya. Artinya ketika terdapat beberapa elemen $A[i]$ yang memiliki nilai $(A[i]\Mod{M})$ sama, sudah pasti lebih menguntungkan untuk hanya menggunakan nilai $A[i]$ yang terkecil. Berdasarkan observasi ini, kita dapat mengurangi banyaknya elemen pada array $A$ menjadi paling banyak $M$ elemen. Terapkan pula hal serupa untuk array $B$.

Kini array $A$ dan array $B$ memiliki paling banyak $M$ elemen. Coba semua kemungkinan pasangan dan cari yang menghasilkan nilai $f$ terkecil. Kompleksitas solusi $\bigO{M^2}$.

\subsection*{Versi Sulit}
\addcontentsline{toc}{subsection}{Versi Sulit} % for pdf indexing
Batasan: $1 \le M \le 100.000$

Untuk nilai elemen $A$ dan $B$ yang cukup besar, perkalian antar suatu elemen array $A$ dengan array $B$ memiliki kontribusi kepada nilai $f$ yang jauh lebih besar daripada bagian penjumlahan dengan suatu elemen array $C$. Artinya nilai elemen $C$ manapun yang dipilih kurang signifikan pada nilai akhir $f$.

Dengan sifat tersebut dan perilaku pemilihan elemen $C$ yang polanya sulit ditebak, kita dapat menganggap penjumlahan dengan suatu elemen $C$ sebagai suatu \textit{error} atau \textit{noise}. Tuliskan rumus $f$ menjadi:
$$f(i,j) = A[i] \times B[j] + err$$

Dengan $err$ adalah suatu nilai yang berkisar antara $0$ sampai dengan $\maxValue$.

Sekarang misalkan kita memilih suatu $i=x$. Kita dapat mencoba seluruh kemungkinan nilai elemen $B$, dari yang paling kecil hingga suatu elemen yang cukup besar sehingga $err$ menjadi tidak lagi signifikan. 

Misalkan $B_{min}$ adalah elemen terkecil pada array $B$. Misalkan juga $C_{min}$ dan $C_{max}$ menyatakan elemen terkecil dan terbesar pada array $C$. Kita harus mencoba seluruh elemen $B[j]$ yang memenuhi:
\begin{align}
A[x] \times B[j] + C_{min} &< A[x] \times B_{min} + C_{max} \nonumber \\
A[x] \times B[j] - A[x] \times B_{min} &< C_{max} - C_{min} \nonumber\\
A[x] (B[j] - B_{min}) &< 1.000.000 \nonumber \\
B[j] - B_{min} &< \frac{1.000.000}{A[x]} \label{bound}
\end{align}

Pada awalnya, $A[x] \times B_{min} + err$ merupakan kandidat solusi. Seluruh $B[j]$ yang tidak memenuhi pertidaksamaan \ref{bound} tidak mungkin menjadi kandidat solusi, karena sudah pasti lebih besar daripada $A[x] \times B_{min} + err$, tanpa peduli berapapun nilai $err$.

Dengan asumsi array $A$ dan array $B$ memiliki nilai yang unik, banyaknya $B[j]$ unik yang perlu diperiksa untuk suatu $A[x]$ dibatasi oleh $O\left(\frac{1.000.000}{A[x]}\right)$. Banyaknya pasangan yang perlu diperiksa dapat diperkirakan dengan cara berikut:
\begin{align}
\sum_{i=0}^{|A|-1} \frac{1.000.000}{A[i]} &= 1.000.000 \sum_{i=0}^{|A|-1} \frac{1}{A[i]} \nonumber \\
&\le 1.000.000 \sum_{i=0}^{\maxValue} \frac{1}{i} \label{eq1} \\
&\le 1.000.000 (\ln{\maxValue} + 1) \label{eq2}
\end{align}

Pada kenyataannya, $|A|$ terbesar adalah $100.000$, yaitu $\frac{1}{10}$ dari $\maxValue$. Jadi pertidaksamaan yang ditunjukkan pada \ref{eq1} merupakan estimasi yang berlebihan. Sebagai catatan, perpindahan dari \ref{eq1} ke \ref{eq2} memanfaatkan sifat \href{https://en.wikipedia.org/wiki/Harmonic_series_%28mathematics%29}{deret harmonik}.

Sebagai kesimpulan, solusi yang digunakan adalah:
\begin{enumerate}
\item Urutkan array $B$ dari kecil ke besar, lalu buang elemen-elemen duplikat sehingga didapatkan array yang unik.
\item Coba seluruh kemungkinan $A[i]$.
\item Untuk suatu $A[i]$ yang dipilih, coba seluruh elemen $B$ mulai dari yang terkecil sampai ke elemen terbesar yang masih memenuhi pertidaksamaan \ref{bound}.
\item Cetak yang menghasilkan nilai $f$ terkecil.
\end{enumerate}

Estimasi kasar berlebihan untuk kompleksitas solusi ini adalah $\bigO{\maxValue \ln{\maxValue}}$.

\end{document}
